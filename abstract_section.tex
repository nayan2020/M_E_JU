
\chapter*{\centering Abstract}

\addcontentsline{toc}{chapter}{Abstract}
Large language models (LLMs) have revolutionized the landscape of natural language processing by enabling advanced reasoning, generation, and task orchestration. While regular expressions (RegEx) remain a staple for pattern-based information extraction, they suffer from limitations in flexibility, contextual understanding, and adaptability. RegEx patterns are rigid, often break with minor format changes, and require extensive domain-specific knowledge to write and maintain—making them unsuitable for dynamic or ambiguous data environments. In contrast, Retrieval-Augmented Generation (RAG) introduces semantic reasoning and contextual adaptation by leveraging relevant knowledge from large corpora. This thesis proposes a novel framework that replaces traditional RegEx with a system built on RAG and Agentic AI. Here, autonomous agents act as orchestrators: selecting appropriate tools, retrieving relevant data, and applying tailored logic based on user intent. The system takes natural language queries, interprets them using retrieved knowledge, and dynamically composes extraction logic that is more robust and adaptable than handcrafted expressions. Our experiments demonstrate that this approach not only generalizes better than static RegEx rules but also adapts to diverse data sources and evolving information needs. This work introduces a paradigm shift in information extraction, opening pathways for intelligent, context-aware automation.